% !TEX encoding = UTF-8
% !TEX TS-program = pdflatex
% !TEX root = ../tesi.tex

%**************************************************************
\chapter{Introduzione}
\label{cap:introduzione}
%**************************************************************
%**************************************************************
\section{L'azienda}

L'attività di stage è stata svolto presso l'azienda iVoIT S.r.l. (logo in figura \ref{fig:Logo-iVoIT}) con sede a Pavoda presso il 
centro direzionale La Cittadella. 
\begin{figure}[!h]
    
    \centering
    \includegraphics[width=0.9\columnwidth]{logo-ivoxit.png} 
    \caption{Logo aziendale iVoIT}
    \label{fig:Logo-iVoIT} 
\end{figure}
\textit{iVoxIT S.r.l.} con \textit{Athesys S.r.l.} e \textit{Monokee S.r.l.} fa parte di un gruppo di aziende fondato nel 2010 dall’unione di professionisti dell’  \gls{itg}\glsfirstoccur (IT) con l’obiettivo di fornire consulenza ad alto livello tecnologico e progettuale. Tra le altre cose, Athesys S.r.l fornisce supporto
nell’istanziazione del processo di \gls{iamg}\glsfirstoccur (IAM) , con particolare
attenzione alla sicurezza nella conservazione e nell’esposizione dei dati sensibili gestiti.
L’azienda opera in tutto il territorio nazionale, prevalentemente nel Nord Italia e
vanta esperienze a livello europeo in paesi quali Olanda, Regno Unito e Svizzera.
Grazie all’adozione delle \gls{bestpracticesg}\glsfirstoccur definite dalle linee guida \gls{itilg}\glsfirstoccur (ITIL)  e alla certificazione ISO 9001 il gruppo è
in grado di assicurare un’alta qualità professionale.

%**************************************************************
\section{L'idea}

Nell’ottica di estendere le funzionalità del prodotto \textit{Monokee} di \gls{iamg} basato su \textit{cloud}, lo stage ha visto lo sviluppo di due moduli applicativi in ambito \gls{blockchaing}\glsfirstoccur.
Il primo modulo è un’applicazione mobile (\textbf{Wallet}) contenente l’identità digitale dell’utente finale mentre il secondo modulo è un \textit{layer} applicativo (\textbf{Service Provider}) per gestire gli accessi alle applicazioni di terze parti.
In figura \ref{fig:diag-mod} un’immagine dei moduli da implementare e il loro posizionamento in un tipico scenario di accesso ai servizi.
\begin{figure}[!h]
    
    \centering
    \includegraphics[width=0.9\columnwidth]{diagrammaComponenti.png} 
    \caption{Diagramma Moduli}
    \label{fig:diag-mod} 
\end{figure}
La tipologia di \gls{blockchaing} da integrare è stata individuata in una prima fase di analisi.

%**************************************************************
\section{Organizzazione del testo}

\begin{description}
    \item[{\hyperref[cap:processi-metodologie]{Il secondo capitolo}}] fornisce una breve introduzione alle metodologie adottate durante lo svolgimento delle attività.
    
    \item[{\hyperref[cap:descrizione-stage]{Il terzo capitolo}}] presenta una serie di approfondimenti relativi al dominio applicativo e tecnologico in cui si colloca il progetto.
    
    \item[{\hyperref[cap:analisi-requisiti]{Il quarto capitolo}}] espone lo studio dei requisiti svolto.
    
    \item[{\hyperref[cap:progettazione-codifica]{Il quinto capitolo}}] descrive la progettazione dei vari componenti presenti nel progetto. Inoltre per alcuni di essi ne approfondisce alcuni aspetti relativi all'implementazione.
    
    \item[{\hyperref[cap:verifica-validazione]{Il sesto capitolo}}] approfondisce le varie attività svolte nel contesto della verifica e della validazione.
    
    \item[{\hyperref[cap:conclusioni]{Nel settimo capitolo}}] fornisce alcune considerazioni personale relative allo stage.
\end{description}

Riguardo la stesura del testo, relativamente al documento sono state adottate le seguenti convenzioni tipografiche:
\begin{itemize}
	\item gli acronimi, le abbreviazioni e i termini ambigui o di uso non comune menzionati vengono definiti nel glossario, situato alla fine del presente documento;
	\item per la prima occorrenza dei termini riportati nel glossario viene utilizzata la seguente nomenclatura: \emph{parola}\glsfirstoccur;
	\item i termini in lingua straniera o facenti parti del gergo tecnico sono evidenziati con il carattere \emph{corsivo}.
\end{itemize}