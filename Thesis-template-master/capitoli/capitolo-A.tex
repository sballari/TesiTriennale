% !TEX encoding = UTF-8
% !TEX TS-program = pdflatex
% !TEX root = ../tesi.tex

%**************************************************************
\chapter{Appendice A}
\section{Strumenti utili per lavorare su Ethereum}
\label{cap:str_eth}
Di seguito una breve descrizione di alcuni strumenti utili per lavorare su \emph{Ethereum}.
\begin{itemize}
    \item \textbf{Truffle}: è una suite di development e testing. Permette di compilare, buildare ed effettuare la migrazione degli SmartContract. Inoltre ha funzioni di debugging e di scripting. La suite offre la possibilità di effettuare test degli SmartContract sia in Javascript (con l’utilizzo di Chai), sia in Solidity. Si riporta di seguito il sito del progetto: \cite{site:truffle}.
    \item \textbf{Ganache}: è uno strumento rapido che permette di creare e mantenere in locale una rete blockchain Ethereum personale. Può essere usata per eseguire test, eseguire comandi e per operazioni di controllo dello stato mentre il codice esegue. Si riporta di seguito il sito del progetto: \cite{site:ganache}.
    \item \textbf{Mist}: è un browser sviluppato direttamente dal team Ethereum in grado di operare transazioni direttamente nella blockchain senza la necessità di possedere un intero nodo. È estremamente immaturo e non utilizzabile in produzione. Si riporta di seguito il sito del progetto: \cite{site:mist}.
    \item \textbf{Parity}: è un client Ethereum che permette di operare sulla rete senza necessità di possedere un intero nodo. Questa soluzione a differenza di Mist dovrebbe risultare più facilmente integrabile nel prodotto senza che l’utente ne abbi consapevolezza. Si riporta di seguito il sito del progetto: \cite{site:parity} .
    \item \textbf{Metamask}: è uno plugin disponibile per i browser Chrome, Firefox, e Opera. Permette di interfacciarsi alla rete Ethereum senza la necessità di eseguire in intero nodo della rete. Il plugin include un wallet con cui l’utente può inserire il proprio account tramite la chiave privata. Una volta inserito l’account il plugin farà da tramite tra l’applicazione e la rete.
    Metamask è utilizzato dalla maggioranza delle applicazioni Ethereum presenti on line, questo però rappresenterebbe un componente esterno compatibile con pochi browser desktop. Si riporta di seguito il sito del progetto: \cite{site:metamask} .
    \item \textbf{Status}: è un progetto che propone una serie di \gls{apig} che permettono di sviluppare un’applicazione mobile nativa operante direttamente su blockchain senza la necessità di possedere un intero nodo. Il sito del progetto propone una serie di applicazioni che utilizzano Status. Tuttavia nessuna di queste applicazioni risulta attualmente rilasciate in nessuno store. Status risulta in early access ed è disponibile per Android e iOS. Il sito del progetto è il seguente: \cite{site:status}.
    \item \textbf{Microsoft Azure}: “Ethereum Blockchain as a Service” è un servizio fornito da Microsoft e ConsenSys che permette di sviluppare a basso costo in un ambiente di dev/test/produzione. Permette di creare reti private, pubbliche e di consorzio. Queste reti saranno poi accessibili attraverso la rete privata Azure. Questa tecnologia rende facile l’integrazione con Cortana Analytics, Power BI, Azure Active Directory, O365 e CRMOL.
\end{itemize}
%**************************************************************





