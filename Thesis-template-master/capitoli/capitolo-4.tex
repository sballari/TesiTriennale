% !TEX encoding = UTF-8
% !TEX TS-program = pdflatex
% !TEX root = ../tesi.tex

%**************************************************************
\chapter{Analisi dei requisiti IW}
\label{cap:analisi-requisiti}
%**************************************************************

\intro{Breve introduzione al capitolo}\\
Questo capitolo ha lo scopo di fornire una definizione dei requisiti individua per la creazione del prodotto Identity Wallet (IW). Le metodologie usate sono tratte dal capitolo quattro di \cite{som:swe}.
Più in particolare la presente capitolo si prefigge di: 
\begin{itemize}
    \item individuare le fonti per la deduzione dei requisiti; 
    \item dedurre i requisiti dalle fonti; 
    \item descrivere i requisiti individuati; 
    \item catalogare i requisiti individuati; 
    \item prioritizzare i requisiti individuati;
\end{itemize}
\section{Specifiche in Linguaggio Naturale}
Il linguaggio naturale ha un’enorme potenza espressiva ma, essendo inerentemente ambiguo, può portare ad incomprensioni. È quindi necessario limitarne l’utilizzo e standardizzarlo, in modo da ridurre al minimo le possibili ambiguità. È comunque fondamentale evitare di utilizzare espressioni e acronimi che possano essere fraintendibili dagli stakeholders, a tal proposito in fondo al documento è presente una lista degli acronimi utilizzato.
\section{Specifiche in Linguaggio Strutturato}
Il linguaggio strutturato mantiene gran parte dell’espressività del linguaggio naturale, fornendo però uno standard schematico che permette l’uniformità della descrizione dei vari requisiti. Sebbene l’utilizzo di un linguaggio strutturato permetta di organizzare i requisiti in modo più ordinato e comprensibile, talvolta la ridotta espressività rende difficile la definizione di requisiti complessi. A tal proposito è possibile integrare la specifica in linguaggio strutturato con una descrizione in linguaggio naturale.
\section{Specifiche in Linguaggio UML Use Case}
Per la definizione dei diagrammi UML dei casi d’uso, viene utilizzato lo standard UML 2.0 \footcite{site:uml}. Nei diagrammi dei casi d’uso vengono mostrati gli attori coinvolti in un’interazione con il sistema in modo schematico, indicando i nomi delle parti coinvolte. Eventuali informazioni aggiuntive possono essere espresse testualmente.


\section{Casi d'uso}

Per lo studio dei casi di utilizzo del prodotto sono stati creati dei diagrammi.
I diagrammi dei casi d'uso (in inglese \emph{Use Case Diagram}) sono diagrammi di tipo \gls{uml} dedicati alla descrizione delle funzioni o servizi offerti da un sistema, così come sono percepiti e utilizzati dagli attori che interagiscono col sistema stesso.

\subsection{Descrizione Attori}
I tipi di attori principali che andranno ad interagire direttamente con il sistema sono essenzialmente tre: 
\begin{itemize}
    \item utente;
    \item utente non registrato;
    \item utente autenticato. 
\end{itemize}   
Tra di essi è presente una relazione di generalizzazione che vede l’attore utente come generalizzazione degli attori utente non registrato e utente registrato. Questo tipo di generalizzazione viene rappresentata graficamente in figura \ref{fig:ger-actors}.
\begin{figure}[!h]
    
    \centering
    \includegraphics[width=0.5\columnwidth]{usecase/use-case-iw/actors.png} 
    \caption{Gerarchia utenti user case}
    \label{fig:ger-actors} 
\end{figure}
Sono stati individuati i seguenti attori secondari: ITF, MonoKee.
\subsubsection{Attori principali}
\begin{itemize}
    \item \textbf{Utente}: l’attore utente è un fruitore generico del sistema. Potrebbe avere o non avere effettuato l’accesso all’applicazione. Da lui derivano gli attori utente non registrato e utente autenticato.	
    \item \textbf{Utente non registrato}: l’attore utente non registrato è una particolare specializzazione dell’attore utente. Unica sua caratteristica è quella di non essere riconosciuto come utente di MonoKee.
    \item \textbf{Utente autenticato}: l’attore utente autenticato è una particolare specializzazione dell’attore utente. Rappresenta un utente che ha effettuato l’accesso al sistema e che è stato riconosciuto all’interno del sistema MonoKee.
\end{itemize}
      
\subsubsection{Attori secondari}
\begin{itemize}
    \item \textbf{ITF}: è il componente dell’estensione che ha il compito di conservare e convalidare tutte le informazioni provenienti dall’IW.
    \item \textbf{MonoKee}: è il componente centrare dell’attuale servizio MonoKee. Ha il compito di fornire le informazioni di accesso del servizio MonoKee. 
\end{itemize}
    


\subsection{Elenco casi d'uso}


\subsubsection{UC1: Azioni utente generico}
\begin{figure}[!htbp] 
    \centering 
    \includegraphics[width=0.7\columnwidth]{usecase/use-case-iw/UC1-azioni-utente.png} 
    \caption{Use Case - UC1: Azioni utente generico}
\end{figure}

\paragraph{Descrizione}  L’utente può visualizzare le informazioni sull’applicazione 
\paragraph{Attore primario}  Utente
\paragraph{Attore secondario}  Nessuno
\paragraph{Precondizioni}  L’utente ha avviato l’applicazione
\paragraph{Postcondizioni}  L’utente ha eseguito le azioni che desiderava compiere in relazione alle sue possibilità
\paragraph{Scenario principale}  
    \begin{enumerate}
        \item UC1.1 Visualizza info applicazione
    \end{enumerate}
\paragraph{Scenari alternativi}  Nessuno


\subsubsection{UC1.1 – Visualizza info applicazione}
\begin{figure}[!htbp] 
    \centering 
    \includegraphics[width=0.7\columnwidth]{usecase/use-case-iw/UC1-1-Visualizza-info-applicazione.png} 
    \caption{Use Case - UC1.1 – Visualizza info applicazione}
\end{figure}

\paragraph{Descrizione}  Il sistema deve visualizzare le informazioni relative all’applicazione mobile e al servizio MonoKee  
\paragraph{Attore primario}  Utente
\paragraph{Attore secondario}  Nessuno
\paragraph{Precondizioni}  L’utente ha avviato l’applicazione
\paragraph{Postcondizioni}  L’utente ha visualizzato le informazioni che desirava riguardo l’applicazione
\paragraph{Scenario principale}  
    \begin{enumerate}
        \item UC1.1.1 Visualizza info applicazione
        \item UC1.1.2 Visualizza info servizio MonoKee
        \item UC1.1.3 Visualizza tutorial utilizzo
    \end{enumerate}
\paragraph{Scenari alternativi}  Nessuno


\subsubsection{UC1.1.1 – Visualizza info app mobile}
\paragraph{Descrizione}  Il sistema deve visualizzare le informazioni tecniche relative all’applicazione mobile
\paragraph{Attore primario}  Utente
\paragraph{Attore secondario}  Nessuno
\paragraph{Precondizioni}  L’utente ha avviato l’applicazione ed ha richiesto la visualizzazione delle informazioni tecniche relative all’applicazione mobile
\paragraph{Postcondizioni}  L’utente ha visualizzato le informazioni con le informazioni tecniche relative all’applicazione mobile
\paragraph{Scenario principale}  
L’utente visualizza un messaggio con le informazioni tecniche relative all’applicazione mobile
\paragraph{Scenari alternativi}  Nessuno


\subsubsection{UC1.1.2 – Visualizza info servizio MonoKee}
\paragraph{Descrizione}  Il sistema deve visualizzare le informazioni relative al servizio MonoKee
\paragraph{Attore primario}  Utente
\paragraph{Attore secondario}  Nessuno
\paragraph{Precondizioni}  L’utente ha avviato l’applicazione ed ha richiesto la visualizzazione delle informazioni relative al servizio MonoKee
\paragraph{Postcondizioni}  L’utente ha visualizzato le informazioni con le informazioni relative al servizio MonoKee
\paragraph{Scenario principale}  
L’utente visualizza un messaggio con le informazioni relative al servizio MonoKee
\paragraph{Scenari alternativi}  Nessuno


\subsubsection{UC1.1.3 – Visualizza tutorial utilizzo}
\paragraph{Descrizione}  Il sistema deve visualizzare un tutorial su come utilizzare l’applicazione IW
\paragraph{Attore primario}  Utente
\paragraph{Attore secondario}  Nessuno
\paragraph{Precondizioni}  L’utente ha avviato l’applicazione ed ha richiesto la visualizzazione di un tutorial su come utilizzare l’applicazione IW
\paragraph{Postcondizioni}  L’utente ha visualizzato il tutorial su come utilizzare l’applicazione IW
\paragraph{Scenario principale}  
L’utente visualizza un tutorial su come utilizzare l’applicazione IW
\paragraph{Scenari alternativi}  Nessuno


\subsubsection{UC2 – Azioni utente non registrato}
\begin{figure}[!htbp] 
    \centering 
    \includegraphics[width=0.7\columnwidth]{usecase/use-case-iw/UC2-azioni-utente-non-registrato.png} 
    \caption{Use Case - UC2: Azioni utente non registrato}
\end{figure}

\paragraph{Descrizione}  L’utente non registrato può eseguire le operazioni di registrazione e accesso al servizio MonoKee 
\paragraph{Attore primario}  Utente non registrato
\paragraph{Attore secondario}  MonoKee
\paragraph{Precondizioni}  L’utente ha avviato l’applicazione ed non è ancora riconosciuto nel sistema
\paragraph{Postcondizioni}  L’utente ha eseguito le azioni che desiderava compiere in relazione alla condizione di non essere registrato
\paragraph{Scenario principale}  
    \begin{enumerate}
        \item UC2.1 Registrazione
        \item UC2.2 Accesso MonoKee
    \end{enumerate}
\paragraph{Scenari alternativi}  
    \begin{enumerate}
        \item l’utente ha fornito dati di registrazione non validi o il doppio inserimento della password non coincide: UC2.3 Visualizzazione messaggio di errore registrazione.
        \item l’utente ha fornito username e password non corrispondenti ha nessun utente registrato al servizio: UC2.4 Visualizzazione messaggio di errore autenticazione.
    \end{enumerate}


\subsubsection{UC2.1 – Registrazione}
\begin{figure}[!htbp] 
    \centering 
    \includegraphics[width=0.7\columnwidth]{usecase/use-case-iw/UC2-1-Registrazione.png} 
    \caption{Use Case - UC2.1: Registrazione}
\end{figure}

\paragraph{Descrizione}  L’utente non registrato può eseguire l’operazione di registrazione 
\paragraph{Attore primario}  Utente non registrato
\paragraph{Attore secondario}  MonoKee
\paragraph{Precondizioni}  L’utente ha avviato l’applicazione, non è ancora riconosciuto nel sistema ed ha espresso la volontà di effettuare la registrazione al servizio MonoKee
\paragraph{Postcondizioni}  L’utente ha eseguito l’operazione di registrazione al sistema
\paragraph{Scenario principale}  
    \begin{enumerate}
        \item UC2.1.1 Inserimento username
        \item UC2.1.2 Inserimento password
        \item UC2.1.3 Reinserimento password
    \end{enumerate}
\paragraph{Scenari alternativi}  Nessuno



\subsubsection{UC2.1.1 – Inserimento username}
\paragraph{Descrizione}  L’utente non registrato deve inserire un username per l’operazione di registrazione
\paragraph{Attore primario}  Utente non registrato
\paragraph{Attore secondario}  Nessuno
\paragraph{Precondizioni}  L’utente ha avviato l’applicazione, non è ancora riconosciuto nel sistema ed il sistema richiede l’inserimento di un username per l’operazione di registrazione
\paragraph{Postcondizioni}  L’utente ha inserito l’username per la registrazione
\paragraph{Scenario principale}  
L’utente non registrato inserisce una stringa tramite l’utilizzo di una text box
\paragraph{Scenari alternativi}  Nessuno


\subsubsection{UC2.1.2 – Inserimento password}
\paragraph{Descrizione}  L’utente non registrato deve inserire una password per l’operazione di registrazione
\paragraph{Attore primario}  Utente non registrato
\paragraph{Attore secondario}  Nessuno
\paragraph{Precondizioni}  L’utente ha avviato l’applicazione, non è ancora riconosciuto nel sistema ed il sistema richiede l’inserimento di una password per l’operazione di registrazione
\paragraph{Postcondizioni}  L’utente ha inserito la password per la registrazione
\paragraph{Scenario principale}  
L’utente non registrato inserisce una stringa tramite l’utilizzo di una text box
\paragraph{Scenari alternativi}  Nessuno



\subsubsection{UC2.1.3 – Reinserimento password}
\paragraph{Descrizione}  L’utente non registrato deve reinserire la password per l’operazione di registrazione
\paragraph{Attore primario}  Utente non registrato
\paragraph{Attore secondario}  Nessuno
\paragraph{Precondizioni}  L’utente ha avviato l’applicazione, non è ancora riconosciuto nel sistema ed il sistema richiede il reinserimento di una password per l’operazione di registrazione
\paragraph{Postcondizioni}  L’utente ha reinserito la password per la registrazione
\paragraph{Scenario principale}  
L’utente non registrato inserisce una stringa tramite l’utilizzo di una text box
\paragraph{Scenari alternativi}  Nessuno



\subsubsection{UC2.2 – Accesso MonoKee}
\begin{figure}[!htbp] 
    \centering 
    \includegraphics[width=0.7\columnwidth]{usecase/use-case-iw/UC2-2-Accesso-MonoKee.png} 
    \caption{Use Case - UC2.1: Accesso MonoKee}
\end{figure}

\paragraph{Descrizione}  L’utente non registrato può eseguire l’operazione di autenticazione 
\paragraph{Attore primario}  Utente non registrato
\paragraph{Attore secondario}  MonoKee
\paragraph{Precondizioni}  L’utente ha avviato l’applicazione, non è ancora riconosciuto nel sistema ed ha espresso la volontà di effettuare l’autenticazione al servizio MonoKee
\paragraph{Postcondizioni}  L’utente ha eseguito l’operazione di accesso al sistema ed è quindi ora riconosciuto come utente autenticato
\paragraph{Scenario principale}  
    \begin{enumerate}
        \item UC2.1.1 Inserimento username
        \item UC2.1.2 Inserimento password
    \end{enumerate}
\paragraph{Scenari alternativi}  Nessuno


\subsubsection{UC2.2.1 – Inserimento username}
\paragraph{Descrizione}  L’utente non registrato deve inserire un username per l’operazione di autenticazione
\paragraph{Attore primario}  Utente non registrato
\paragraph{Attore secondario}  Nessuno
\paragraph{Precondizioni}  L’utente ha avviato l’applicazione, non è ancora riconosciuto nel sistema ed il sistema richiede l’inserimento di un username per l’operazione di autenticazione
\paragraph{Postcondizioni}  L’utente ha inserito l’username per l’autenticazione
\paragraph{Scenario principale}  
L’utente non registrato inserisce una stringa tramite l’utilizzo di una text box
\paragraph{Scenari alternativi}  Nessuno



\subsubsection{UC2.2.2 – Inserimento password}
\paragraph{Descrizione}  L’utente non registrato deve inserire una password per l’operazione di autenticazione
\paragraph{Attore primario}  Utente non registrato
\paragraph{Attore secondario}  Nessuno
\paragraph{Precondizioni}  L’utente ha avviato l’applicazione, non è ancora riconosciuto nel sistema ed il sistema richiede l’inserimento di una password per l’operazione di autentificazione
\paragraph{Postcondizioni}  L’utente ha inserito la password per l’autentificazione
\paragraph{Scenario principale}  
L’utente non registrato inserisce una stringa tramite l’utilizzo di una text box
\paragraph{Scenari alternativi}  Nessuno



\subsubsection{UC2.3 – Visualizza messaggio di errore registrazione}
\paragraph{Descrizione}  L’utente non registrato fornisce username già esistente o il doppio inserimento della password non coincide
\paragraph{Attore primario}  Utente non registrato
\paragraph{Attore secondario}  Nessuno
\paragraph{Precondizioni}  L’utente ha avviato l’applicazione, non è ancora riconosciuto nel sistema ed il sistema ha inserito un username già esistente o delle password non coincidenti durante la registrazionerichiede l’inserimento di una password per l’operazione di autentificazione
\paragraph{Postcondizioni}  L’utente ha visualizzato un messaggio di errore relativo all’impossibilità di effettuare la registrazione con i dati forniti
\paragraph{Scenario principale}  
L’utente visualizza un messaggio di errore relativo all’impossibilità di effettuare la registrazione con i dati forniti
\paragraph{Scenari alternativi}  Nessuno



\subsubsection{UC2.4 – Visualizza messaggio di errore autenticazione}
\paragraph{Descrizione}  L’utente non registrato fornisce username e password che non corrispondono a nessun utente registrato al servizio MonoKee
\paragraph{Attore primario}  Utente non registrato
\paragraph{Attore secondario}  Nessuno
\paragraph{Precondizioni}  L’utente ha avviato l’applicazione, non è ancora riconosciuto nel sistema ed il sistema ha inserito un username e una password che non corrispondono a nessun utente registrato al servizio MonoKee
\paragraph{Postcondizioni}  L’utente ha visualizzato un messaggio di errore relativo all’impossibilità di effettuare l’autenticazione
\paragraph{Scenario principale}  
L’utente visualizza un messaggio di errore relativo all’impossibilità di effettuare l’autenticazione
\paragraph{Scenari alternativi}  Nessuno



\subsubsection{UC3 – Azioni utente autenticato}
\begin{figure}[!htbp] 
    \centering 
    \includegraphics[width=0.7\columnwidth]{usecase/use-case-iw/UC3-Azioni-utente-autententicato.png} 
    \caption{Use Case - UC3: Azioni utente autenticato}
\end{figure}

\paragraph{Descrizione}  L’utente autenticato può eseguire le operazioni legate alla gestione della sua identità e alla presentazione dei propri dati ad un SP
\paragraph{Attore primario}  Utente Autenticato
\paragraph{Attore secondario}  ITF
\paragraph{Precondizioni}  L’utente ha avviato l’applicazione ed è riconosciuto nel sistema come utente di MonoKee
\paragraph{Postcondizioni}  L’utente ha eseguito le azioni che desiderava compiere in relazione alla condizione essere riconosciuto come utente di MonoKee
\paragraph{Scenario principale}  
    \begin{enumerate}
        \item UC3.1 Visualizza QR dell’informazione certificata
        \item UC3.2 Visualizza chiave pubblica
        \item UC3.3 Visualizza chiave privata
        \item UC3.4 Inserimento informazione personale
        \item UC3.5 Visualizza lista certificazioni
        \item UC3.6 Elimina certificazione
    \end{enumerate}
\paragraph{Scenari alternativi}  Nessuno





\subsubsection{UC3.1 – Visualizza QR dell’informazione certificata}
\paragraph{Descrizione}  L’utente autenticato può visualizzare nel proprio schermo un codice QR che rappresenta un’informazione certificata
\paragraph{Attore primario}  Utente Autenticato
\paragraph{Attore secondario}  Nessuno
\paragraph{Precondizioni}  L’utente ha avviato l’applicazione, è riconosciuto nel sistema come utente di MonoKee e ha richiesto di visualizzare il codice QR di una certificazione precedentemente inserita.
\paragraph{Postcondizioni}  L’utente ha visualizzato il codice QR che rappresenta la certificazione selezionata
\paragraph{Scenario principale}  
L’utente seleziona e poi visualizza il codice QR che rappresenta la certificazione selezionata
\paragraph{Scenari alternativi}  Nessuno


\subsubsection{UC3.2 – Visualizza chiave pubblica}
\paragraph{Descrizione}  L’utente autenticato può visualizzare la chiave pubblica generata al momento della registrazione
\paragraph{Attore primario}  Utente Autenticato
\paragraph{Attore secondario}  Nessuno
\paragraph{Precondizioni}  L’utente ha avviato l’applicazione, è riconosciuto nel sistema come utente di MonoKee e ha richiesto la visualizzazione della chiave pubblica.
\paragraph{Postcondizioni}  L’utente ha visualizzato la propria chiave pubblica precedentemente generata
\paragraph{Scenario principale}  
L’utente visualizza la propria chiave pubblica precedentemente generata
\paragraph{Scenari alternativi}  Nessuno



\subsubsection{UC3.2 – Visualizza chiave privata}
\paragraph{Descrizione}  L’utente autenticato può visualizzare la chiave privata generata al momento della registrazione
\paragraph{Attore primario}  Utente Autenticato
\paragraph{Attore secondario}  Nessuno
\paragraph{Precondizioni}  L’utente ha avviato l’applicazione, è riconosciuto nel sistema come utente di MonoKee e ha richiesto la visualizzazione della chiave privata.
\paragraph{Postcondizioni}  L’utente ha visualizzato la propria chiave privata precedentemente generata
\paragraph{Scenario principale}  
L’utente visualizza la propria chiave privata precedentemente generata
\paragraph{Scenari alternativi}  Nessuno



\subsubsection{UC3.4 – Inserimento informazione personale}
\begin{figure}[!htbp] 
    \centering 
    \includegraphics[width=0.7\columnwidth]{usecase/use-case-iw/UC3-4-Inserimento-informazione-personale.png} 
    \caption{Use Case - UC3.4: Inserimento informazione personale}
\end{figure}

\paragraph{Descrizione}  L’utente autenticato può inserire una certificazione e sottometterla all’ITF
\paragraph{Attore primario}  Utente Autenticato
\paragraph{Attore secondario}  ITF
\paragraph{Precondizioni}  L’utente ha avviato l’applicazione, è riconosciuto nel sistema come utente di MonoKee, e ha intende inserire una nuova certificazione alla propria identità
\paragraph{Postcondizioni}  L’utente ha inserito la certificazione e questa è stata presentata all’ITF
\paragraph{Scenario principale}  
    \begin{enumerate}
        \item UC3.4.1 Inserimento nome certificazione
        \item UC3.4.2 Inserimento descrizione certificazione
        \item UC3.4.3 Visualizza resoconto
    \end{enumerate}
\paragraph{Scenari alternativi}  Nessuno



\subsubsection{UC3.4.1 – Inserimento nome certificazione}
\paragraph{Descrizione}  L’utente autenticato deve inserire un nome per l’operazione di inserimento certificazione
\paragraph{Attore primario}  Utente Autenticato
\paragraph{Attore secondario}  Nessuno
\paragraph{Precondizioni}  L’utente ha avviato l’applicazione, è riconosciuto nel sistema ed il sistema richiede l’inserimento di un nome per l’operazione di inserimento certificazione
\paragraph{Postcondizioni}  L’utente ha inserito il nome per l’inserimento della certificazione
\paragraph{Scenario principale}  
L’utente autenticato inserisce una stringa tramite l’utilizzo di una text box
\paragraph{Scenari alternativi}  Nessuno



\subsubsection{UC3.4.2 – Inserimento descrizione certificazione}
\paragraph{Descrizione}  L’utente autenticato deve inserire una descrizione per l’operazione di inserimento certificazione
\paragraph{Attore primario}  Utente Autenticato
\paragraph{Attore secondario}  Nessuno
\paragraph{Precondizioni} L’utente ha avviato l’applicazione, è riconosciuto nel sistema ed il sistema richiede l’inserimento di una descrizione per l’operazione di inserimento certificazione
\paragraph{Postcondizioni}  L’utente ha inserito la descrizione per l’inserimento della certificazione
\paragraph{Scenario principale}  
L’utente autenticato inserisce un insieme di stringhe tramite l’utilizzo di una text box
\paragraph{Scenari alternativi}  Nessuno




\subsubsection{UC3.4.3 – Visualizza resoconto}
\paragraph{Descrizione}  L’utente autenticato può visualizzare un resoconto dei dati inseriti durante la procedura di inserimento certificazione
\paragraph{Attore primario}  Utente Autenticato
\paragraph{Attore secondario}  Nessuno
\paragraph{Precondizioni} L’utente ha avviato l’applicazione, è riconosciuto nel sistema come utente di MonoKee, ha iniziato una procedura di inserimento certificazione e ha richiesto la visualizzazione del resoconto dei dati inseriti
\paragraph{Postcondizioni}  L’utente ha visualizzato un resoconto dei dati inseriti durante la procedura di inserimento certificazione
\paragraph{Scenario principale}  
L’utente visualizza un resoconto dei dati inseriti durante la procedura di inserimento certificazione
\paragraph{Scenari alternativi}  Nessuno






\subsubsection{UC3.5 – Visualizza lista certificazioni}
\begin{figure}[!htbp] 
    \centering 
    \includegraphics[width=0.7\columnwidth]{usecase/use-case-iw/UC3-5-Visualizza-lista-certificazioni.png} 
    \caption{Use Case - UC3.5: Visualizza lista certificazioni}
\end{figure}

\paragraph{Descrizione}  L’utente autenticato può visualizzare una lista con il nome e l’identificativo della certificazione associate alla propria identità
\paragraph{Attore primario}  Utente Autenticato
\paragraph{Attore secondario}  Nessuno
\paragraph{Precondizioni}  L’utente ha avviato l’applicazione, è riconosciuto nel sistema come utente di MonoKee e ha richiesto la visualizzazione della lista delle certificazioni
\paragraph{Postcondizioni}  L’utente ha visualizzato la lista delle certificazioni
\paragraph{Scenario principale}  
    \begin{enumerate}
        \item UC3.5.1 Visualizza singola certificazione
    \end{enumerate}
\paragraph{Scenari alternativi}  Nessuno





\subsubsection{UC3.5.1 – Visualizza singola certificazione}
\begin{figure}[!htbp] 
    \centering 
    \includegraphics[width=0.7\columnwidth]{usecase/use-case-iw/UC3-5-1-Visualizza-singola-certificazione.png} 
    \caption{Use Case - UC3.5.1: Visualizza singola certificazione}
\end{figure}

\paragraph{Descrizione}  L’utente autenticato può visualizzare i dettagli di una certificazione selezionata della lista delle certificazioni
\paragraph{Attore primario}  Utente Autenticato
\paragraph{Attore secondario}  ITF
\paragraph{Precondizioni}  L’utente ha avviato l’applicazione, è riconosciuto nel sistema come utente di MonoKee e ha richiesto la visualizzazione di una specifica entry della lista delle certificazioni
\paragraph{Postcondizioni}  L’utente ha visualizzato i dettagli di una specifica certificazione della lista
\paragraph{Scenario principale}  
    \begin{enumerate}
        \item UC3.5.1.1 Visualizza nome certificazione
        \item UC3.5.1.2 Visualizza descrizione certificazione
        \item UC3.5.1.3 Visualizza stato certificazione
    \end{enumerate}
\paragraph{Scenari alternativi}  Nessuno




\subsubsection{UC3.5.1.1 – Visualizza nome certificazione}
\paragraph{Descrizione}  L’utente autenticato può visualizzare il nome di una certificazione selezionata della lista delle certificazioni
\paragraph{Attore primario}  Utente Autenticato
\paragraph{Attore secondario}  Nessuno
\paragraph{Precondizioni} L’utente ha avviato l’applicazione, è riconosciuto nel sistema come utente di MonoKee e ha richiesto la visualizzazione del nome di una specifica entry della lista delle certificazioni
\paragraph{Postcondizioni}  L’utente ha visualizzato il nome di una specifica certificazione della lista
\paragraph{Scenario principale}  
L’utente visualizza il nome di una specifica certificazione della lista
\paragraph{Scenari alternativi}  Nessuno




\subsubsection{UC3.5.1.2 – Visualizza descrizione certificazione}
\paragraph{Descrizione}  L’utente autenticato può visualizzare la descrizione di una certificazione selezionata della lista delle certificazioni
\paragraph{Attore primario}  Utente Autenticato
\paragraph{Attore secondario}  Nessuno
\paragraph{Precondizioni} L’utente ha avviato l’applicazione, è riconosciuto nel sistema come utente di MonoKee e ha richiesto la visualizzazione della descrizione di una specifica entry della lista delle certificazioni
\paragraph{Postcondizioni}  L’utente ha visualizzato la descrizione di una specifica certificazione della lista
\paragraph{Scenario principale}  
L’utente visualizza la descrizione di una specifica certificazione della lista
\paragraph{Scenari alternativi}  Nessuno



\subsubsection{UC3.5.1.3 – Visualizza stato certificazione}
\paragraph{Descrizione}  L’utente autenticato può visualizzare lo stato di una certificazione selezionata della lista delle certificazioni. L’informazione proviene dall’ITF.
\paragraph{Attore primario}  Utente Autenticato
\paragraph{Attore secondario}  ITF
\paragraph{Precondizioni} L’utente ha avviato l’applicazione, è riconosciuto nel sistema come utente di MonoKee e ha richiesto la visualizzazione dello stato di una specifica entry della lista delle certificazioni
\paragraph{Postcondizioni}  L’utente ha visualizzato lo stato di una specifica certificazione della lista
\paragraph{Scenario principale}  
L’utente visualizza una stringa che può essere confermata da un TTP o non confermata.
\paragraph{Scenari alternativi}  Nessuno



\subsubsection{UC3.6 – Elimina certificazione}
\paragraph{Descrizione}  L’utente autenticato può eliminare una certificazione selezionata
\paragraph{Attore primario}  Utente Autenticato
\paragraph{Attore secondario}  ITF
\paragraph{Precondizioni} L’utente ha avviato l’applicazione, è riconosciuto nel sistema come utente di MonoKee e ha richiesto l’eliminazione di una specifica certificazione
\paragraph{Postcondizioni}  La certificazione non è più presente dal sistema e pure dall’ITF
\paragraph{Scenario principale}  
L’utente seleziona e poi esprime la volontà di eliminare la certificazione certificata
\paragraph{Scenari alternativi}  Nessuno

















\newpage
\section{Tracciamento dei requisiti}

\subsection{Fonti}
Per la deduzione dei requisiti utente e di sistema, che verranno presentati nelle sezioni a seguire, sono stati usati come fonti lo studio Gartner \footcite{farah:The-Dawn-of-Decentralized-Identity}, il capitolo \emph{Studio di fattibilità IW} e gli Use Case presentati nella sezione \emph{Casi d'uso}. La struttura e le convenzioni usate sono ispirate dal capitolo di \cite{som:swe}. In seguito vengono riportate le categorie che vengono usate per la catalogazione:
\begin{itemize}
    \item F: requisito funzionale;
    \item V: requisito di vincolo;
    \item Q: requisito di qualità.
\end{itemize}
    
Per l’attribuzione della priorità viene usata la tecnica MoSCoW, quindi gli indici usati sono i seguenti:
\begin{itemize}
    \item M: must;
    \item S: should; 
    \item C: could; 
    \item W: will.
\end{itemize} 
    
Nelle tabelle \ref{tab:requisiti-funzionali}, \ref{tab:requisiti-qualitativi} e \ref{tab:requisiti-vincolo} sono riassunti i requisiti e il loro tracciamento con gli use case delineati in fase di analisi.

\newpage
\begin{table}%
\caption{Tabella del tracciamento dei requisti funzionali}
\label{tab:requisiti-funzionali}
\begin{tabularx}{\textwidth}{lXl}
\hline\hline
\textbf{Codice} & \textbf{Descrizione} & \textbf{Fonte}\\
\hline
R[F][C]0001     & Il sistema potrebbe permettere ad un utente di visualizzare le informazioni dell’applicazione & UC1, UC1.1 \\
\hline
R[F][C]0002     & Il sistema potrebbe permettere di visualizzare le info tecniche dell’applicazione & UC1.1.2 \\
\hline
R[F][C]0003     & Il sistema potrebbe permettere di visualizzare una descrizione del servizio MonoKee & UC1.1.2 \\
\hline
R[F][C]0004     & Il sistema potrebbe permettere di visualizzare un tutorial esplicativo sul suo utilizzo & UC1.1.3 \\
\hline
R[F][M]0005     & Il sistema deve permettere di potersi registrare al servizio & UC2, UC2.1 \\
\hline
R[F][M]0006     & Il sistema deve permettere di essere riconosciuto dal sistema MonoKee & UC2, UC2.2 \\
\hline
R[F][M]0007     & Il sistema deve visualizzare un messaggio di errore in caso i dati forniti durante la registrazione non dovessero essere validi & UC2, UC2.3 \\
\hline
R[F][M]0008     & Il sistema deve visualizzare un messaggio di errore in caso i dati forniti durante la procedura di autenticazione non dovessero essere corretti & UC2, UC2.4 \\
\hline
R[F][M]0009     & Il sistema deve permettere di inserire uno username nell’ottica della procedura di registrazione  & UC2.1.1\\
\hline
R[F][M]0010    & Il sistema deve permettere di inserire una password nell’ottica della procedura di registrazione & UC2.1.2 \\
\hline
R[F][M]0011     & Il sistema deve permettere di reinserire la password nell’ottica della procedura di registrazione & UC2.1.3 \\
\hline
R[F][M]0012     & Il sistema deve permettere di inserire uno username nell’ottica della procedura di autentificazione & UC2.2.1 \\
\hline
R[F][M] 0013     & Il sistema deve permettere di inserire una password nell’ottica della procedura di autentificazione & UC2.2.2 \\
\hline
R[F][M] 0014     & Il sistema deve permettere ad un utente autenticato di poter generare un codice QR di un certificato inserito nel sistema & UC3, UC3.1 \\
\hline
R[F][M] 0015     & Il sistema deve permettere ad un utente autenticato di visualizzare la chiave pubblica & UC3, UC3.2 \\
\hline
R[F][M] 0016     & Il sistema deve permettere ad un utente autenticato di visualizzare la chiave privata & UC3, UC3.3 \\
\hline
R[F][M] 0017     & Il sistema deve permettere ad un utente autenticato di inserire un’informazione personale & UC3, UC3.4 \\
\hline
R[F][M] 0018     & Il sistema deve permettere ad un utente autenticato di visualizzare una lista di certificazioni associate alla propria identità & UC3, UC3.5 \\
\hline
R[F][M] 0019     & Il sistema deve permettere ad un utente autenticato di eliminare una certificazione associata alla propria identità & UC3, UC3.6 \\
\hline
R[F][M] 0020     & Il sistema deve permettere ad un utente autenticato di inserire il nome della certificazione nel contesto dell’inserimento di certificazione & UC3.4.1 \\
\hline
R[F][M] 0021     & Il sistema deve permettere ad un utente autenticato di una descrizione della certificazione nel contesto dell’inserimento di una certificazione & UC3.4.2 \\
\hline
R[F][M] 0022     & Il sistema deve permettere ad un utente autenticato di visualizzare un resoconto dei dati inseriti durante la procedura di inserimento certificato & UC3.4.3 \\
\hline
R[F][M] 0023     & Il sistema deve permettere ad un utente autenticato di visualizzare i dettagli di una singola certificazione & UC3.5.1 \\
\hline
R[F][M] 0024     & Il sistema deve permettere ad un utente autenticato di visualizzare il nome di una certificazione esistente & UC3.5.1.1 \\
\hline
R[F][M] 0025     & Il sistema deve permettere ad un utente autenticato di visualizzare la certificazione di una certificazione esistente & UC3.5.1.2 \\
\hline
R[F][S] 0026     & Il sistema dovrebbe permettere ad un utente autenticato di visualizzare lo stato di una certificazione esistente & UC3.5.1.3 \\
\hline
\end{tabularx}
\end{table}%



\begin{table}%
\caption{Tabella del tracciamento dei requisiti di vincolo}
\label{tab:requisiti-vincolo}
\begin{tabularx}{\textwidth}{lXl}
\hline\hline
\textbf{Codice} & \textbf{Descrizione} & \textbf{Fonte}\\
\hline
R[V][M] 0027    & Il sistema deve offrire le proprie funzionalità come applicazione mobile &  IW Studio di fattibilità \\
\hline
R[V][M] 0028    & Il sistema è implementato tramite l’uso di Xamarin & IW Studio di fattibilità \\
\hline
R[V][M] 0029    & Il progetto prevede almeno i seguenti quattro ambienti di sviluppo: Local, Test, Staging, Production & IW Studio di fattibilità \\
\hline
R[V][M] 0030    & Il prodotto è sviluppato utilizzando uno strumento di linting & IW Studio di fattibilità \\
\hline
R[V][M] 0031    & Il sistema deve mantenere la chiave privata sempre in locale & IW Studio di fattibilità \\
\hline
\end{tabularx}
\end{table}%



\begin{table}%
    \caption{Tabella del tracciamento dei requisiti qualitativi}
    \label{tab:requisiti-qualitativi}
    \begin{tabularx}{\textwidth}{lXl}
    \hline\hline
    \textbf{Codice} & \textbf{Descrizione} & \textbf{Fonte}\\
    \hline
    R[Q][S] 0032    & Il progetto prevede un ragionevole set di test di unità e di test di integrazione & - \\
    \hline
    R[Q][S] 0033   & I test possono essere eseguiti localmente o come parte di integrazione continua & - \\
    \hline
    R[Q][S] 0034    & Il sistema solo alla fine sarà testato nel network pubblico di prova & - \\
    \hline
    R[Q][S] 0035    & Il codice sorgente del prodotto e la documentazione necessaria per l’utilizzo sono versionati in repository pubblici usando GitHub, BitBucket o GitLab & - \\
    \hline
    R[Q][C] 0036    & Lo sviluppo si eseguirà utilizzano un approccio incrementale  & IW Studio di fattibilità \\
    \hline
    \end{tabularx}
    \end{table}%


\begin{table}%
    \caption{Tabella del tracciamento dei requisiti con le fonti}
    \label{tab:requisiti-fonte}
    \begin{tabularx}{\textwidth}{lX}
    \hline\hline
    \textbf{Codice}  & \textbf{Fonte}\\
    \hline
    R[F][C]0001    & UC1, UC1.1  \\
    \hline
    R[F][C]0002   & UC1.1.2  \\
    \hline
    R[F][C]0003    & UC1.1.2  \\
    \hline
    R[F][C]0004    & UC1.1.3  \\
    \hline
    R[F][M]0005    & UC2, UC2.1  \\
    \hline
    R[F][M]0006    & UC2, UC2.2  \\
    \hline
    R[F][M]0007    & UC2, UC2.3  \\
    \hline
    R[F][M]0008    & UC2, UC2.4  \\
    \hline
    R[F][M]0009    & UC2.1.1  \\
    \hline
    R[F][M]0010    & UC2.1.2  \\
    \hline
    R[F][M]0011    & UC2.1.3  \\
    \hline
    R[F][M]0012    & UC2.2.1  \\
    \hline
    R[F][M] 0013    & UC2.2.2  \\
    \hline
    R[F][M] 0014    & UC3, UC3.1  \\
    \hline
    R[F][M] 0015    & UC3, UC3.2  \\
    \hline
    R[F][M] 0016    & UC3, UC3.3  \\
    \hline
    R[F][M] 0017    & UC3, UC3.4  \\
    \hline
    R[F][M] 0018    & UC3, UC3.5  \\
    \hline
    R[F][M] 0019    & UC3, UC3.6  \\
    \hline
    R[F][M] 0020    & UC3.4.1  \\
    \hline
    R[F][M] 0021    & UC3.4.2  \\
    \hline
    R[F][M] 0022    & UC3.4.3  \\
    \hline
    R[F][M] 0023    & UC3.5.1  \\
    \hline
    R[F][M] 0024    & UC3.5.1.1  \\
    \hline
    R[F][M] 0025    & UC3.5.1.2  \\
    \hline
    R[F][S] 0026    & UC3.5.1.3  \\
    \hline
    R[V][M] 0027    & IW Studio di fattibilità  \\
    \hline
    R[V][M] 0028    & IW Studio di fattibilità  \\
    \hline
    R[V][M] 0029    & IW Studio di fattibilità  \\
    \hline
    R[V][M] 0030    & IW Studio di fattibilità  \\
    \hline
    R[V][M] 0031    & IW Studio di fattibilità  \\
    \hline
    R[Q][S] 0032    & -  \\
    \hline
    R[Q][S] 0033    & -  \\
    \hline
    R[Q][S] 0034    & -  \\
    \hline
    R[Q][S] 0035    & -  \\
    \hline
    R[Q][C] 0036    & IW Studio di fattibilità  \\
    \hline
    \end{tabularx}
    \end{table}%

\begin{table}%
    \caption{Tabella del tracciamento delle fonti con i requisiti}
    \label{tab:fonte-req}
    \begin{tabularx}{\textwidth}{lX}
    \hline\hline
    \textbf{Fonte}  & \textbf{Codice}\\
    \hline
    UC1    & R[F][C]0001  \\
    \hline
    UC1.1   & R[F][C]0001  \\
    \hline
    UC1.1.2    & R[F][C]0002  \\
    \hline
    UC1.1.2    & R[F][C]0003  \\
    \hline
    UC2   & R[F][M]0005, R[F][M]0006, R[F][M]0007, R[F][M]0008 \\
    \hline
    UC2.1    & R[F][M]0005  \\
    \hline
    UC2.2    & R[F][M]0006  \\
    \hline
    UC2.3    & R[F][M]0007  \\
    \hline
    UC2.4 & R[F][M]0008 \\
    \hline
    UC2.1.1 & R[F][M]0009 \\
    \hline
    UC2.1.2 & R[F][M]0010 \\
    \hline
    UC2.1.3 & R[F][M]0011 \\
    \hline
    UC2.2.1 & R[F][M]0012 \\
    \hline
    UC2.2.2 & R[F][M] 0013 \\
    \hline
    UC3 & R[F][M] 0014, R[F][M] 0015, R[F][M] 0016, R[F][M] 0017, R[F][M] 0018, R[F][M] 0019\\
    \hline
    UC3.1 & R[F][M] 0014 \\
    \hline
    UC3.2 & R[F][M] 0015 \\
    \hline
    UC3.3 & R[F][M] 0016 \\
    \hline
    UC3.4 & R[F][M] 0017 \\
    \hline
    UC3.5 & R[F][M] 0018 \\
    \hline
    UC3.6 & R[F][M] 0019 \\
    \hline
    UC3.4.1 & R[F][M] 0020 \\
    \hline
    UC3.4.2 & R[F][M] 0021 \\
    \hline
    UC3.4.3 & R[F][M] 0022 \\
    \hline
    UC3.5.1 & R[F][M] 0023 \\
    \hline
    UC3.5.1.1 & R[F][M] 0024 \\
    \hline
    UC3.5.1.2 & R[F][M] 0025 \\
    \hline
    UC3.5.1.3 & R[F][S] 0026 \\
    \hline
    IW Studio di fattibilità & R[V][M] 0027, R[V][M] 0028, R[V][M] 0029, R[V][M] 0030, R[V][M] 0031, R[Q][C] 0036 \\
    \hline
    - & R[Q][S] 0032, R[Q][S] 0033, R[Q][S] 0034, R[Q][S] 0035 \\
    \hline
    \end{tabularx}
    \end{table}%