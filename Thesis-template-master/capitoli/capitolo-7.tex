% !TEX encoding = UTF-8
% !TEX TS-program = pdflatex
% !TEX root = ../tesi.tex

%**************************************************************
\chapter{Conclusioni}
\label{cap:conclusioni}
%**************************************************************

%**************************************************************

%**************************************************************
%**************************************************************
\section{Conoscenze acquisite}
Il progetto si colloca in un servizio più grande qual è \emph{Monokee}, motivo per cui si è reso neccessario uno sviluppo che tenesse conto di differenti progetti esistenti, di cui alcuni già in produzione ed altri in via di sviluppo (i.e. il componente ITF). Questo mi ha permesso di lavorare in un contesto che richiedeva di interagire costantemente con altri team e quindi operare in maniera controllata, disciplinata e coordinata. È stato quindi fondamentale acquisire pratiche di ingegneria del software e attuare correttamente le pratiche aziendali (queste si basavano su \gls{scrumg}).

Il piano di lavoro prevedeva lo sviluppo di due componenti: l'\textbf{Identity Wallet} (IW) e il \textbf{Service Provider} (SP). Questi sono da considerarsi due prodotti completamente separati tra di loro che concorrono, in aggiunta al componente ITF (sviluppato da un altro stagista), a fornire un unico servizio. Sono applicativi di natura diversa; il primo un'applicazione mobile, il secondo un'applicazione server, che, di conseguenza hanno richiesto conoscenze diverse dandomi la possibilità di acquisire più competenze.

Di seguito viene proposta una lista delle principali competente acquisite durante le attività di stage e non precedentemente conosciute:
\begin{itemize}
    \item apprendimento del linguaggio C\#;
    \item sviluppo di interfacce touch in Xamarin;
    \item creazione di servizi RESTful;
    \item apprendimento dell'uso di WebSocket;
    \item uso del framework .NET Core;
    \item uso del framework  Asp.NET;
    \item progettazione di un'architettura Event Driven;
    \item uso di RabbitMQ e MassTransit;
    \item integrazione con SAML.
\end{itemize}

Inoltre ho avuto l'opportunità di approfondire e mettere in pratica alcune delle conoscenze acquisite durante il corso di laurea triennale; tra queste riporto:
\begin{itemize}
    \item metodologie e pratiche di ingegneria del software;
    \item uso di HTML5 e javascript;
    \item uso di lambda funzioni (\emph{arrow function}) in un linguaggio imperativo;
    \item ideazione e codifica di test.
\end{itemize}


%**************************************************************
\section{Valutazione personale}
L'esperienza svolta presso \emph{IvoxIT} è stata di grande valore a conclusione del mio percorso di studi per approfondire e ampliare le mie conoscenze e acquisire delle competenze, ma soprattutto per avere uno spaccato di quello che potrebbe essere il mondo del lavoro inerente al corso di studi fatto. ''''Tra queste reputo fondamentale, prima fra tutte, la capacità di relazionarsi con persone gerarchicamente superiori. Questa esperienza, infatti, mi ha fatto capire come molto spesso non si tratti di avere le giuste competenze, ma di saper mostrare quello che si ha.

Ho avuto modo inoltre di lavorare ed inserirmi in un gruppo estremamente affiatato e coeso con cui ho avuto la possibilità di confrontarmi al fine di rendere il mio progetto veramente utile agli scopi aziendali. Questa stretta relazione mi ha permesso di poter fare affidamento su persone con grande esperienza e capacità e rendere di conseguenza più veloci e di maggior qualità le attività di analisi, progettazione e codifica. Ha aprezzato come le persone con cui ho lavorato, anche se non a conoscenza della particolare tecnologia, disponessero di attitudine e metodo, cosa che permetteva loro un rapido apprendimento e la capacità di fornirmi un aiuto concreto. I rapporti instaurati non sono stati solo di natura prettamente didattica e lavorativa, ma anche di amicizia e stima reciproca. Penso di aver avuto modo di lavorare e consultarmi con persone di elavata caratura sia lavorativa che personale. 

Il progetto è stato sviluppato nei tempi previsti e nonostante le grandi fluttuazioni in termini di requisiti e di aspetti progettuali che questo ha avuto, posso ritenere il risultato soddisfacente. Tuttavia il prodotto finale ha fatto emergere le innumerevoli problematiche legate all'uso della tecnologia \gls{blockchaing}. È infatti emerso come l'uso di una blockchain permissionless abbia dei tempi di elaborazione difficilmente accettabili da un utente medio. Nel corso della settimana finale si sono svolti test di prestazione sulla rete \gls{ropsteng} che hanno mostrato come anche per le chiamate più semplici, che richiedevano almeno un'operazione di scrittura, ci fosse un tempo minimo di attesa pari a 30s. Ciò nonostante, a seguito di operazioni di ottimizzazione, siamo riusciuti ad eseguire le chiamate alla blockchain in parallelo arrivando ad un tempo totale necessario al login pari a 32s. Tale lentezza è dovuta alla necessaria \emph{proof of interest}, che in una rete di questo tipo consiste nel risolvere un problema matematico (\emph{proof of work}) di difficoltà variabile in base alle necessità.  
In conclusione, ritengo che i vantaggi legati all'uso di una blockchain siano:
\begin{itemize}
    \item affidabilità;
    \item disponibilità;
    \item eliminazione di intermediari
\end{itemize} 
non sufficienti a giustificare nella maggioranza dei casi una durata così elevata per una semplice operazione di login.
Inoltre, uno degli obiettivi principali dell'azienda era quello di inglobale questo sistema nell'attuale. Questo ha portato ad una serie di scelte che annulato alcune fondamentali caratteristiche della blockchain. Il componente SP rappresenta un sistema centralizzato e quindi un grosso \emph{point of failure} che mina l'elevata disponibilità tipica della blockchain (in quanto sistema distribuito). Un altro punto critico è rappresentato dall'uso di account gestiti direttamente dall'azienda per tutte le operazioni. Questa scelta aveva lo scopo di rendere gli utenti liberi dal doversi gestire i propri account, ma in una previsione di alto utilizzo del sistema renderebbe difficile per una persona verificare le transazioni. Quindi farebbe venir meno la carrateristica fiducia tipica delle blockchain, in quanto la mole di transazioni renderebbe difficile l'identificazione delle proprie.

Lo stage mi ha permesso di mettere in pratica molte conoscenze acquisite durante il triennio, tra queste alcune di carattere tecnologico, altre di carattere metodologico. Ritengo siano stati di grandissimo aiuto i vari corsi di programmazione e il corso di ingegneria del software. Oltre a raffinare conoscenze già in mio possesso ho avuto modo di apprendere tecnologie e modi di lavorare nuovi che saranno estrememente utili nel corso delle mie esperienze future.

Al netto delle precedentemente citate conoscenze ritengo di aver acquisito qualcosa di molto più importante, la capacità di analizzare un problema in maniera analita e trovarne una soluzione adatta senza farsi limitare dalle tecnologie e dalle limitate conoscenze in possesso. Sarà sempre possibile apprendere nuove tecniche e metodi. Le capacità analitiche e di ragionamento sono utili in qualsiasi situazione e permettono di adattarsi al meglio alle difficoltà. Queste non diventano obselete nel corso degli anni ed una volta imparate sono per sempre.

In conclusione ritengo che l'esperienza svolta sia stata fondamentale per il mio futuro sia come informatico, sia come lavoratore. Questi mesi mi hanno spronato a migliore le mie competenza e le mia capacità, inoltre mi hanno portato a cambiare la mia attidutine verso al lavoro. Lo stage ha rafforzato la mai convinzione nel proseguire gli studi con la laurea magistrale in modo tale da potermi presentare fra due anni con un bagaglio più completo verso il mondo del lavoro.