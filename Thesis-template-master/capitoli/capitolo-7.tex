% !TEX encoding = UTF-8
% !TEX TS-program = pdflatex
% !TEX root = ../tesi.tex

%**************************************************************
\chapter{Conclusioni}
\label{cap:conclusioni}
%**************************************************************

%**************************************************************

%**************************************************************
%**************************************************************
\section{Conoscenze acquisite}
Il progetto si colloca in un servizio più grande qual è \emph{Monokee}, motivo per cui si è reso neccessario uno sviluppo che tenesse conto di differenti progetti esistenti, di cui alcuni già in produzione ed altri in via di sviluppo (i.e. il componente ITF). Questo mi ha permesso di lavorare in un contesto che richiedeva di interagire costantemente con altri team e quindi operare in maniera controllata, disciplinata e coordinata. È stato quindi fondamentale acquisire pratiche di ingegneria del software e attuare correttamente le pratiche aziendali (queste si basavano su \gls{scrumg}).

Il piano di lavoro prevedeva lo sviluppo di due componenti: l'\textbf{Identity Wallet} (IW) e il \textbf{Service Provider} (SP). Questi sono da considerarsi due prodotti completamente separati tra di loro che concorrono, in aggiunta al componente ITF (sviluppato da un altro stagista), a fornire un unico servizio. Sono applicativi di natura diversa; il primo un'applicazione mobile, il secondo un'applicazione server, che, di conseguenza hanno richiesto conoscenze diverse dandomi la possibilità di acquisire più competenze.

Di seguito viene proposta una lista delle principali competente acquisite durante le attività di stage e non precedentemente conosciute:
\begin{itemize}
    \item apprendimento del linguaggio C\#;
    \item sviluppo di interfacce touch in Xamarin;
    \item creazione di servizi RESTful;
    \item apprendimento dell'uso di WebSocket;
    \item uso del framework .NET Core;
    \item uso del framework  Asp.NET;
    \item progettazione di un'architettura Event Driven;
    \item uso di RabbitMQ e MassTransit;
    \item integrazione con SAML.
\end{itemize}

Inoltre ho avuto l'opportunità di approfondire e mettere in pratica alcune delle conoscenze acquisite durante il corso di laurea triennale; tra queste riporto:
\begin{itemize}
    \item metodologie e pratiche di ingegneria del software;
    \item uso di HTML5 e javascript;
    \item uso di lambda funzioni (\emph{arrow function}) in un linguaggio imperativo;
    \item ideazione e codifica di test.
\end{itemize}


%**************************************************************
\section{Valutazione personale}
Il periodo di stage svolto presso \emph{IvoxIT}, a conclusione del mio percorso di studi, è stato fondamentale per approfondire ed ampliare le conoscenze in mio possesso; tra queste, alcune di carattere tecnologico, altre di carattere metodologico e, a tal proposito, indispensabili sono stati i vari corsi di programmazione e il corso di ingegneria del software. Oltre a ciò, ho avuto modo di sviluppare nuove competenze, apprendere tecnologie e sistemi che mi torneranno utili nel futuro. 

In questi mesi ho avuto l'occasione di lavorare ed inserirmi in un gruppo estremamente affiatato e coeso con cui ho avuto la possibilità di confrontarmi al fine di rendere il mio progetto veramente utile agli scopi aziendali. Ho potuto fare affidamento su persone con grande esperienza e capacità, e questo mi ha permesso di rendere più veloci e di maggior qualità le attività di analisi, progettazione e codifica. Ho inoltre aprezzato come i miei colleghi, anche se non a conoscenza della particolare tecnologia, disponessero di attitudine e metodo, cosa che permetteva loro un rapido apprendimento e la capacità di fornirmi un aiuto concreto. I rapporti instaurati non sono stati solo di natura prettamente didattica e lavorativa, ma anche di amicizia e stima reciproca. Penso di aver avuto modo di lavorare e consultarmi con persone di elavata caratura sia lavorativa che personale. 

Il progetto è stato sviluppato nei tempi previsti e nonostante le grandi fluttuazioni in termini di requisiti e di aspetti progettuali che questo ha avuto, il risultato è stato soddisfacente. Tuttavia il prodotto finale ha fatto emergere le innumerevoli problematiche legate all'uso della tecnologia \gls{blockchaing}. Una blockchain permissionless ha infatti tempi di elaborazione difficilmente accettabili da un utente medio. Nel corso della settimana finale si sono svolti test di prestazione sulla rete \gls{ropsteng} che hanno mostrato come anche per le chiamate più semplici, che richiedevano almeno un'operazione di scrittura, ci fosse un tempo minimo di attesa pari a 30s. Ciò nonostante, a seguito di operazioni di ottimizzazione, siamo riusciuti ad eseguire le chiamate alla blockchain in parallelo arrivando ad un tempo totale necessario al login pari a 32s. Tale lentezza è dovuta alla necessaria \emph{proof of interest}, che in una rete di questo tipo consiste nel risolvere un problema matematico (\emph{proof of work}) di difficoltà variabile in base alle esigenza.  
In conclusione, ritengo che i vantaggi legati all'uso di una blockchain siano:
\begin{itemize}
    \item affidabilità;
    \item disponibilità;
    \item eliminazione di intermediari
\end{itemize} 
non sufficienti a giustificare nella maggioranza dei casi una durata così elevata per una semplice operazione di login.
Inoltre, uno degli obiettivi principali dell'azienda era quello di inglobale tale sistema in quello attuale. Ciò ha portato ad una serie di scelte che hanno annulato alcune fondamentali caratteristiche della blockchain. Il componente SP rappresenta un sistema centralizzato e quindi un grosso \emph{point of failure} che mina l'elevata disponibilità tipica della blockchain (in quanto sistema distribuito). Un altro punto critico riguarda invece l'uso di account gestiti direttamente dall'azienda per tutte le operazioni. Lo scopo era quello di rendere gli utenti liberi dal dover gestire i propri account, ma in una previsione di alto utilizzo del sistema renderebbe complicato per una persona verificare le transazioni e quindi farebbe venir meno la carrateristica di fiducia tipica delle blockchain, in quanto la mole di transazioni renderebbe difficile l'identificazione delle proprie.

In conclusione da questa esperienza ho appreso che una buona preparazione sia in termini di conoscenze che di competenze, è sì necessaria, ma non preclusiva ai fini del lavoro. È invece indispensabile l'attitudine e il pensiero analitico, il quale permette di trovare la soluzione più adatta di fronte ai problemi e affrontare al meglio le varie difficoltà 

Infine, avendo avuto uno spaccato reale di quello che il mio percorso di studi si può riversare a livello professionale, ho avuto maggiore chiarezza dei possibili ambiti occupazionali e di conseguenza ho potuto fare delle considerazioni, rafforzando così la mia convinzione nel proseguire con gli studi magistrali e maturando prospettive più precise.