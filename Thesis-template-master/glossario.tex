
%**************************************************************
% Acronimi
%**************************************************************
\renewcommand{\acronymname}{Acronimi e abbreviazioni}

\newacronym[description={\glslink{apig}{Application Program Interface}}]
    {api}{API}{Application Program Interface}

\newacronym[description={\glslink{umlg}{Unified Modeling Language}}]
    {uml}{UML}{Unified Modeling Language}

\newacronym[description={\glslink{itg}{Information Tecnology}}]
    {it}{IT}{Information Tecnology}

\newacronym[description={\glslink{iamg}{Identity Access Management}}]
    {iam}{IAM}{Identity Access Management}

\newacronym[description={\glslink{pclg}{Portable Class Libraries}}]
    {pcl}{PCL}{Portable Class Libraries}

\newacronym[description={\glslink{itilg}{Information
Technology Infrastructure Library}}]
    {itil}{ITIL}{Information
    Technology Infrastructure Library}

\newacronym[description={\glslink{srpg}{Single Responsability Principle}}]
    {srp}{SRP}{Single Responsability Principle}
%**************************************************************
% Glossario
%**************************************************************
%\renewcommand{\glossaryname}{Glossario}

\newglossaryentry{apig}
{
    name=\glslink{api}{API},
    text=Application Program Interface,
    sort=api,
    description={in informatica con il termine \emph{Application Programming Interface API} (ing. interfaccia di programmazione di un'applicazione) si indica ogni insieme di procedure disponibili al programmatore, di solito raggruppate a formare un set di strumenti specifici per l'espletamento di un determinato compito all'interno di un certo programma. La finalità è ottenere un'astrazione, di solito tra l'hardware e il programmatore o tra software a basso e quello ad alto livello semplificando così il lavoro di programmazione}
}

\newglossaryentry{umlg}
{
    name=\glslink{uml}{UML},
    text=UML,
    sort=uml,
    description={in ingegneria del software \emph{UML, Unified Modeling Language} (ing. linguaggio di modellazione unificato) è un linguaggio di modellazione e specifica basato sul paradigma object-oriented. L'\emph{UML} svolge un'importantissima funzione di ``lingua franca'' nella comunità della progettazione e programmazione a oggetti. Gran parte della letteratura di settore usa tale linguaggio per descrivere soluzioni analitiche e progettuali in modo sintetico e comprensibile a un vasto pubblico}
}

\newglossaryentry{itg}
{
    name =\glslink{it}{IT},
    text=Information Tecnology,
    sort=it,
    description={in informatica indica l'utilizzo di elaboratori e attrezzature di telecomunicazione per memorizzare, recuperare, trasmettere e manipolare dati, spesso nel contesto di un'attività commerciale o di un'altra attività economica}
}

\newglossaryentry{iamg}
{
    name =\glslink{iam}{IAM},
    text=Identity Access Management,
    sort=iam,
    description={in informatica indica la disciplina che consente ai giusti individui di accedere alle giuste risorse nel giusto
    momento per le giuste ragioni. L’IAM risponde alla necessità di garantire un
    adeguato accesso alle risorse in ambiti tecnologici sempre più eterogenei.}
}
\newglossaryentry{bestpracticesg}
{
    name = best practise,
    description={Esperienza, procedura o azione più significativa, o comunque che ha permesso di
    ottenere i migliori risultati, relativamente a svariati contesti e obiettivi preposti.}
}
\newglossaryentry{itilg}
{
    name =\glslink{itil}{ITIL},
    text=Information
    Technology Infrastructure Library,
    sort=itil,
    description={In informatica è un insieme di linee guida nella gestione dei servizi IT (IT Service Management) e consiste in una serie di pubblicazioni che forniscono indicazioni sull'erogazione di servizi IT di qualità e sui processi e mezzi necessari a supportarli da parte di una organizzazione.}
}
\newglossaryentry{agileg}
{
    name = agile,
    description={Nell'ingegneria del software, l'espressione metodologia agile (o sviluppo agile del software, in inglese agile software development, abbreviato in ASD) si riferisce a un insieme di metodi di sviluppo del software emersi a partire dai primi anni 2000 e fondati su un insieme di principi comuni, direttamente o indirettamente derivati dai princìpi del \citep{site:agile-manifesto}}
}

\newglossaryentry{scrumg}
{
    name = Scrum,
    description={Scrum è un framework agile per la gestione del ciclo di sviluppo del software, iterativo ed incrementale, concepito per gestire progetti e prodotti software o applicazioni di sviluppo, creato e sviluppato da Ken Schwaber e Jeff Sutherland}
}

\newglossaryentry{blockchaing}
{
	name = {blockchain},
	description = { Struttura dati a catena composta da una serie di blocchi in continua espansione che cresce man mano che nuove
			transazioni vengono confermate come parte di un nuovo blocco. Ogni nuovo blocco viene concatenato al blocco
			precedente della blockchain esistente tramite un sistema crittografico proof-of-work.
		}
}

\newglossaryentry{SmartContractg}
{
	name = {SmartContract},
	description = { Protocolli per computer progettati per facilitare, verificare o imporre la negoziazione o l'esecuzione di un
			contratto.
		}
}

\newglossaryentry{Ethereumg}
{
	name = {Ethereum},
	description = { Piattaforma decentralizzata pubblica ed open-source basata sulla creazione di
			SmartContract. Permette la creazione di applicazioni che operano su blockchain
			in modo che non ci sia alcuna possibilità di downtime, censura, frodi o interferenze
			da terze parti.
		}
}

\newglossaryentry{stakeholderg}
{
	name = {stakeholders},
	description = { Ciascuno dei soggetti direttamente o indirettamente coinvolti in un progetto o nell'attività di un'azienda.
			Fanno parte di questa categoria persone in qualità di fornitori, di committenti o di clienti.
		}
}



\newglossaryentry{pclg}
{
    name =\glslink{pcl}{PCL},
    text= Portable Class Libraries,
    sort=pcl,
    description={PCL è un approccio alla condivisione del codice tra le diverse edizioni dell’app destinate a diversi sistemi operativi mobili sviluppato da Xamarin.}
}


\newglossaryentry{srpg}
{
    name =\glslink{srp}{SRP},
    text= single responsibility principle,
    sort=srp,
    description={Nella programmazione orientata agli oggetti, il principio di singola responsabilità (single responsibility principle, abbreviato con SRP) afferma che ogni elemento di un programma (classe, metodo, variabile) deve avere una sola responsabilità, e che tale responsabilità debba essere interamente incapsulata dall'elemento stesso.}
}


\newglossaryentry{frameworkg}
{
    name = {framework},
    description={Originariamente il framework era concepito come un aggregato di strumenti ben collegati che servono per lavorare in un determinato dominio. Ora invece nell’ambito SW è una libreria architetturale riutilizzabile e generica con la caratteristica di poter essere istanziata in base alle proprie necessità. }
}

\newglossaryentry{suitetestg}
{
    name = {suite di test},
    description={Una collezione di test utili a vericare il comportamento di un determinato programma}
}

\newglossaryentry{ropsteng}
{
    name = {Ropsten},
    description={Ropsten Ethereum, noto anche come “Ethereum Testnet”, è una rete di test che esegue lo stesso protocollo utilizzato da Ethereum e viene utilizzata prima di essere distribuito sulla rete principale (Mainnet).}
}
