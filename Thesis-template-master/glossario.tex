
%**************************************************************
% Acronimi
%**************************************************************
\renewcommand{\acronymname}{Acronimi e abbreviazioni}

\newacronym[description={\glslink{apig}{Application Program Interface}}]
    {api}{API}{Application Program Interface}

\newacronym[description={\glslink{umlg}{Unified Modeling Language}}]
    {uml}{UML}{Unified Modeling Language}

\newacronym[description={\glslink{itg}{Information Tecnology}}]
    {it}{IT}{Information Tecnology}

\newacronym[description={\glslink{iamg}{Identity Access Management}}]
    {iam}{IAM}{Identity Access Management}

\newacronym[description={\glslink{itilg}{Information
Technology Infrastructure Library}}]
    {itil}{ITIL}{Information
    Technology Infrastructure Library}
%**************************************************************
% Glossario
%**************************************************************
%\renewcommand{\glossaryname}{Glossario}

\newglossaryentry{apig}
{
    name=\glslink{api}{API},
    text=Application Program Interface,
    sort=api,
    description={in informatica con il termine \emph{Application Programming Interface API} (ing. interfaccia di programmazione di un'applicazione) si indica ogni insieme di procedure disponibili al programmatore, di solito raggruppate a formare un set di strumenti specifici per l'espletamento di un determinato compito all'interno di un certo programma. La finalità è ottenere un'astrazione, di solito tra l'hardware e il programmatore o tra software a basso e quello ad alto livello semplificando così il lavoro di programmazione}
}

\newglossaryentry{umlg}
{
    name=\glslink{uml}{UML},
    text=UML,
    sort=uml,
    description={in ingegneria del software \emph{UML, Unified Modeling Language} (ing. linguaggio di modellazione unificato) è un linguaggio di modellazione e specifica basato sul paradigma object-oriented. L'\emph{UML} svolge un'importantissima funzione di ``lingua franca'' nella comunità della progettazione e programmazione a oggetti. Gran parte della letteratura di settore usa tale linguaggio per descrivere soluzioni analitiche e progettuali in modo sintetico e comprensibile a un vasto pubblico}
}

\newglossaryentry{itg}
{
    name =\glslink{it}{IT},
    text=Information Tecnology,
    sort=it,
    description={in informatica indica l'utilizzo di elaboratori e attrezzature di telecomunicazione per memorizzare, recuperare, trasmettere e manipolare dati, spesso nel contesto di un'attività commerciale o di un'altra attività economica}
}

\newglossaryentry{iamg}
{
    name =\glslink{iam}{IAM},
    text=Identity Access Management,
    sort=iam,
    description={in informatica indica la disciplina che consente ai giusti individui di accedere alle giuste risorse nel giusto
    momento per le giuste ragioni. L’IAM risponde alla necessità di garantire un
    adeguato accesso alle risorse in ambiti tecnologici sempre più eterogenei.}
}
\newglossaryentry{bestpracticesg}
{
    name = best practise,
    description={Esperienza, procedura o azione più significativa, o comunque che ha permesso di
    ottenere i migliori risultati, relativamente a svariati contesti e obiettivi preposti.}
}
\newglossaryentry{itilg}
{
    name =\glslink{itil}{ITIL},
    text=Information
    Technology Infrastructure Library,
    sort=itil,
    description={In informatica è un insieme di linee guida nella gestione dei servizi IT (IT Service Management) e consiste in una serie di pubblicazioni che forniscono indicazioni sull'erogazione di servizi IT di qualità e sui processi e mezzi necessari a supportarli da parte di una organizzazione.}
}
